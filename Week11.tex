% Options for packages loaded elsewhere
\PassOptionsToPackage{unicode}{hyperref}
\PassOptionsToPackage{hyphens}{url}
%
\documentclass[
]{article}
\usepackage{amsmath,amssymb}
\usepackage{iftex}
\ifPDFTeX
  \usepackage[T1]{fontenc}
  \usepackage[utf8]{inputenc}
  \usepackage{textcomp} % provide euro and other symbols
\else % if luatex or xetex
  \usepackage{unicode-math} % this also loads fontspec
  \defaultfontfeatures{Scale=MatchLowercase}
  \defaultfontfeatures[\rmfamily]{Ligatures=TeX,Scale=1}
\fi
\usepackage{lmodern}
\ifPDFTeX\else
  % xetex/luatex font selection
\fi
% Use upquote if available, for straight quotes in verbatim environments
\IfFileExists{upquote.sty}{\usepackage{upquote}}{}
\IfFileExists{microtype.sty}{% use microtype if available
  \usepackage[]{microtype}
  \UseMicrotypeSet[protrusion]{basicmath} % disable protrusion for tt fonts
}{}
\makeatletter
\@ifundefined{KOMAClassName}{% if non-KOMA class
  \IfFileExists{parskip.sty}{%
    \usepackage{parskip}
  }{% else
    \setlength{\parindent}{0pt}
    \setlength{\parskip}{6pt plus 2pt minus 1pt}}
}{% if KOMA class
  \KOMAoptions{parskip=half}}
\makeatother
\usepackage{xcolor}
\usepackage[margin=1in]{geometry}
\usepackage{graphicx}
\makeatletter
\def\maxwidth{\ifdim\Gin@nat@width>\linewidth\linewidth\else\Gin@nat@width\fi}
\def\maxheight{\ifdim\Gin@nat@height>\textheight\textheight\else\Gin@nat@height\fi}
\makeatother
% Scale images if necessary, so that they will not overflow the page
% margins by default, and it is still possible to overwrite the defaults
% using explicit options in \includegraphics[width, height, ...]{}
\setkeys{Gin}{width=\maxwidth,height=\maxheight,keepaspectratio}
% Set default figure placement to htbp
\makeatletter
\def\fps@figure{htbp}
\makeatother
\setlength{\emergencystretch}{3em} % prevent overfull lines
\providecommand{\tightlist}{%
  \setlength{\itemsep}{0pt}\setlength{\parskip}{0pt}}
\setcounter{secnumdepth}{-\maxdimen} % remove section numbering
\ifLuaTeX
  \usepackage{selnolig}  % disable illegal ligatures
\fi
\IfFileExists{bookmark.sty}{\usepackage{bookmark}}{\usepackage{hyperref}}
\IfFileExists{xurl.sty}{\usepackage{xurl}}{} % add URL line breaks if available
\urlstyle{same}
\hypersetup{
  pdftitle={Spotify Recommender System},
  pdfauthor={Shaya Engelman},
  hidelinks,
  pdfcreator={LaTeX via pandoc}}

\title{Spotify Recommender System}
\author{Shaya Engelman}
\date{2023-11-19}

\begin{document}
\maketitle

\hypertarget{assignment-prompt}{%
\subsection{Assignment Prompt}\label{assignment-prompt}}

\begin{enumerate}
\def\labelenumi{\arabic{enumi}.}
\item
  Perform a Scenario Design analysis as described below. ~Consider
  whether it makes sense for your selected recommender system to perform
  scenario design twice, once for the organization (e.g.~Amazon.com) and
  once for the organization's customers.
\item
  Attempt to reverse engineer what you can about the site, from the site
  interface and any available information that you can find on the
  Internet or elsewhere.
\item
  Include specific recommendations about how to improve the site's
  recommendation capabilities going forward.~
\end{enumerate}

\hypertarget{spotifys-recommender-system}{%
\subsection{Spotify's Recommender
System}\label{spotifys-recommender-system}}

The purpose of this report is to conduct a Scenario Design analysis of
Spotify's recommender system, particularly, the music segment as opposed
to podcasts. Spotify, employs a sophisticated recommendation engine to
enhance user experience and engagement. This analysis aims to understand
the current system, identify stakeholders, and propose recommendations
for both the organization (Spotify) and its customers (users).

Recommender systems are algorithms designed to recommend content to
users. There are a variety of techniques different systems can utilize.
Spotify's recommender system utilizes a combination of collaborative
filtering, content-based filtering, and machine learning algorithms.
Collaborative filtering entails using information gleaned from other
users to recommend content to the user, such as other songs liked by
users who share some liked songs with the user. Content-based filtering
entails analyzing the user's liked content and finding other similar
content on the platform. This analysis involves artist-sourced metadata,
listening to the `feel' of the music (slow, fast etc.) and using natural
language processing (NLP) to analyze the text of the song.

Spotify's primary business goals related to the recommender system
include improving user satisfaction, increasing user engagement, and
obtaining and retaining paid subscribers. The recommender system plays a
crucial role in delivering personalized content to users.

There are multiple parties invested in Spotify's recommender system.
First and most obviously are the listeners. The act of listening to
music has shifted from individuals meticulously selecting a particular
album or song for enjoyment to a more convenient experience of choosing
a playlist and finding satisfaction in the majority of the newly
discovered songs. In order to acheive that, a good recommender system is
necessary. Aside from the listeners, the music creators have an interest
in the systems too. The creators get paid based on the amount of listens
their content gets; a system that introduces their music to potential
fans benefits them as well.

\textbf{Reverse Engineering:}

Exploring the Spotify app interface reveals how recommendations are
presented to users, including recommending similar artists, personalized
playlists, radio stations, and the popular ``Discover'' feature.
Understanding the user experience is vital for proposing improvements.

\begin{figure}
\centering
\includegraphics{images/Screenshot 2023-11-21 212707-02.png}
\caption{Some of the ways Spotify recommends content (I clearly have a
toddler at home)}
\end{figure}

Upon creating a new account, Spotify prompts the user for some favorite
artists and songs. The app also makes it easy to `like' a song. Both of
these features help Spotify create the tailored recommendations.

\textbf{Recommendations:}

\textbf{For Spotify (Organization)}:

\emph{Algorithmic Improvements:}

\begin{itemize}
\item
  Genre and Mood Enhancement: suggest improvements to content-based
  filtering by refining the understanding of music genres, moods, and
  contextual factors.
\item
  User Engagement Features: Propose introducing features that encourage
  user engagement, such as personalized playlists for special occasions
  or collaborative playlists with friends.
\item
  Feedback Mechanism: Implement a direct and user-friendly feedback
  mechanism, enabling users to provide explicit feedback on individual
  songs or playlists.
\end{itemize}

\textbf{For Spotify Users (Customers):}

\begin{itemize}
\item
  Encourage users to provide more detailed preferences, including
  favorite genres, artists, and moods. Allow customization of profiles
  to reflect changes in music taste.
\item
  Artists should clearly label all the metadata of their content, such
  as the genre and label of the song.
\end{itemize}

In conclusion, this Scenario Design analysis of Spotify's recommender
system provides valuable insights for both the organization and its
users. By implementing the recommended improvements, Spotify can enhance
user satisfaction, increase engagement, and maintain its position as a
leading music streaming platform. Regular iterations based on user
feedback and evolving listening patterns will be key to the ongoing
success of the recommender system.

\end{document}
